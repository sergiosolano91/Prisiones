\chapter{Glosario}
\begin{small}
\begin{enumerate}[A]
  \item[]
\begin{verbatim}
    /  see slash marks
    \@ following period ends sentence
    \\[*][extra-space] new line.
    \, thin space, math and text mode
    \; thick space, math mode and text mode
    \: medium space, math mode and text mode
    \! negative thin space, math mode and text mode
    \- hyphenation; tabbing
    \= set tab, see tabbing
    \> tab, see tabbing
    \< back tab, see tabbing
    \+ see tabbing
    \' accent or tabbing
    \` accent or tabbing
    \| double vertical lines, math mode
    \( start math environment
    \) end math environment
    \[ begin displaymath environment
    \] end displaymath environment
\end{verbatim}
  \item[A]
\begin{verbatim}
\addcontentsline{file}{sec_unit}{entry} adds an entry to the specified list
            or table
\addtocontents{file}{text} adds text (or formatting commands) directly to
            the file that generates the specified list or table
\addtocounter{counter}{value} increments the counter
\address{Return address}
\addtolength{len-cmd}{len} increments a length command, see Useful
            Measurement Macros
\addvspace adds a vertical space of a specified height
\alph causes the current value of a specified counter to be printed in
            alphabetic characters
\appendix changes the way sectional units are numbered so that information
            after the command is considered part of the appendix
\arabic causes the current value of a specified counter to be printed
            in Arabic numbers
\author declares the author(s).
\end{verbatim}
  \item[B]
\begin{verbatim}
\backslash  prints a backslash
\baselineskip a length command (see Useful Measurement Macros), which
            specifies the minimum space between the bottom of two
            successive lines in a paragraph
\baselinestretch scales the value of \baselineskip
\bf Boldface typeface
\bibitem generates a labeled entry for the bibliography
\bigskipamount
\bigskip equivalent to \vspace{\bigskipamount}
\boldmath bold font in math mode
\end{verbatim}
  \item[C]
\begin{verbatim}
\cal Calligraphic style in math mode
\caption generate caption for figures and tables
\cdots Centered dots
\centering Used to center align LaTeX environments
\chapter Starts a new chapter.
\circle
\cite Used to make citations from the provided bibliography
\cleardoublepage
\clearpage Ends the current page and causes any floats to be printed.
\cline Adds horizontal line in a table that spans only to a range of cells.
\closing
\copyright makes � sign.
\end{verbatim}
  \item[D]
\begin{verbatim}
\dashbox
\date
\ddots
\documentclass[options]{style} Used to begin a latex document
\dotfill
\end{verbatim}
  \item[E]
\begin{verbatim}
\em Italicizes the text which is inside curly braces with the command. Such
            as {\em This is in italics}. This command allows nesting.
\emph
\ensuremath (LaTeX2e)
\euro Prints euro symbol. Requires eurosym package.
\end{verbatim}
  \item[F]
\begin{verbatim}
\fbox
\flushbottom
\fnsymbol
\footnote Creates a footnote.
\footnotemark
\footnotesize Sets font size.
\footnotetext
\frac
\frame
\framebox Like \makebox but creates a frame around the box.
\frenchspacing
\end{verbatim}
  \item[G]
  \item[H]
\begin{verbatim}
\hfill Abbreviation for \hspace{\fill}.
\hline adds a horizontal line in a tabular environment.
\hrulefill
\hspace Produces horizontal space.
\huge Sets font size.
\Huge Sets font size.
\hyphenation
\end{verbatim}
  \item[I]
\begin{verbatim}
\include
\includegraphics Inserts an image. Requires graphicx package.
\includeonly
\indent
\input Used to read in LaTex files
\it Italicizes the text which is inside curly braces with the command. Such
        as {\it This is in italics}. \em is generally preferred since
        this allows nesting.
\item Creates an item in a list. Used in list structures.
\end{verbatim}
  \item[J]
  \item[K]
\begin{verbatim}
\kill
\end{verbatim}
  \item[L]
\begin{verbatim}
\label Used to create label which can be later referenced with \ref.
\large Sets font size.
\Large Sets font size.
\LARGE Sets font size.
\LaTeX Prints LaTeX logo.
\LaTeXe Prints current LaTeX version logo.
\ldots Prints sequence of three dots.
\left
\lefteqn
\line
\linebreak Suggests LaTeX to break line in this place.
\linethickness
\linewidth
\listoffigures
\listoftables
\location
\end{verbatim}
  \item[M]
\begin{verbatim}
\makebox Defines a box that has a specified width, independent from its content.
\maketitle
\markboth
\markright
\mathcal
\mathop
\mbox
\medskip
\multicolumn
\multiput
\end{verbatim}
  \item[N]
\begin{verbatim}
\newcommand
\newcounter
\newenvironment
\newfont
\newlength
\newline Ends current line and starts a new one.
\newpage Ends current page and starts a new one.
\newsavebox
\newtheorem
\nocite
\noindent
\nolinebreak
\normalsize Sets default font size.
\nopagebreak Suggests LaTeX not to break page in this place.
\not
\end{verbatim}
  \item[O]
  \item[P]
\begin{verbatim}
\pagebreak Suggests LaTeX breaking page in this place.
\pagenumbering
\pageref Used to reference to number of page where a previously declared \label
            is located.
\pagestyle
\par Starts a new paragraph
\paragraph Starts a new paragraph.
\parbox Defines a box whose contents are created in paragraph mode.
\parindent Normal paragraph indentation.
\parskip
\part Starts a new part of a book.
\protect
\providecommand (LaTeX2e)
\put
\end{verbatim}
  \item[Q]
  \item[R]
\begin{verbatim}
\raggedbottom Command used for top justified within other environments.
\raggedleft Command used for right justified within other environments.
\raggedright Command used for left justified within other environments.
\raisebox Creates a box and raises its content.
\ref Used to reference to number of previously declared \label.
\renewcommand
\right
\rm
\roman
\rule Creates a line of specified width and height.
\end{verbatim}
  \item[S]
\begin{verbatim}
\savebox Makes a box and saves it in a named storage bin.
\sbox The short form of \savebox with no optional arguments.
\sc
\scriptsize Sets font size.
\section Starts a new section.
\setcounter
\setlength
\settowidth
\sf
\shortstack
\signature
\sl
\slash See slash marks
\small Sets font size.
\smallskip
\sout Strikes out text. Requires ulem package.
\space force ordinary space
\sqrt Creats a root (default square, but magnitude can be given as an
            optional parameter).
\stackrel Takes two arguments and stacks the first on top of the second.
\subparagraph Starts a new subparagraph.
\subsection Starts a new subsection.
\subsubsection Starts a new sub-subsection.
\end{verbatim}
  \item[T]
\begin{verbatim}
\tableofcontents
\telephone
\TeX Prints TeX logo.
\textbf{} Sets bold font style.
\textit{} Sets italic font style.
\textmd{} Sets medium weight of a font.
\textnormal{} Sets normal font.
\textrm{} Sets roman font family.
\textsc{} Sets font style to small caps.
\textsf{} Sets sans serif font family.
\textsl{} Sets slanted font style.
\texttt{} Sets typewriter font family.
\textup{} Sets upright shape of a font.
\textwidth
\textheight
\thanks
\thispagestyle
\tiny Sets font size.
\title
\today Writes current day.
\tt
\twocolumn
\typeout
\typein
\end{verbatim}
  \item[U]
\begin{verbatim}
\uline Underlines text. Requires ulem package.
\underbrace
\underline
\unitlength
\usebox
\usecounter
\uwave Creates wavy underline. Requires ulem package.
\end{verbatim}
  \item[V]
\begin{verbatim}
\value
\vbox{text} Encloses a paragraph's text to prevent it from running over a
            page break
\vdots Creates vertical dots.
\vector
\verb Creates inline verbatim text.
\vfill
\vline
\vphantom
\vspace
\end{verbatim}
  \item[W]
  \item[X]
  \item[Y]
  \item[Z]
\end{enumerate}
\end{small} 