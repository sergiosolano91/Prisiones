\chapter{Trabajo futuro}
Dentro de los posibles trabajos futuros de proyecci�n de poblaciones carcelarias se encuentran:

Proyecciones m�s precisas se pueden obtener al trabajar con los ingresos, salidas del sistema, duraci�n de las condenas y de los procesos. As� mismo, ser�a posible estimar el impacto que los cambios normativos tienen sobre las din�micas de la poblaci�n carcelaria y la influencia de variables ex�genas como el desempleo o los niveles de pobreza en los diferentes componentes.

Proyectar la tasa de encarcelamiento espec�ficas por edad, g�nero y delito, incluso a nivel regional y departamental resultar�a en un insumo importante para la planificaci�n de infraestructura y la reducci�n del hacinamiento.

La estructura de las tasa de encarcelamiento espec�ficas por edad podr�a permitir el uso de datos funcionales que consideren la interacci�n entre las tasas de encarcelamiento espec�ficas. Trabajar con tasas de encarcelamiento por edad, delito podr�a permitir una mejor comprensi�n de la estructura observada de las tasas de encarcelamiento. 

Aunque en este trabajo no se defini� como parte de los objetivos, es com�n analizar: tasas de reincidencia y primera ofensa. Posteriores proyecciones podr�an ser enriquecidas con esta informaci�n. 

El desarrollo de paquetes similares a MARSS \cite{Holmes2018} , que crean accesos sencillos a paquetes de optimizaci�n de la funci�n de m�xima verosimilitud que permitan imponer restricciones sobre los par�metros a estimar. 