\chapter{Modelos estado-espacio}

\section{Marco te�rico}

\section{Selecci�n de Software}

Dado que el alcance del trabajo es aplicado, se recurre a software predise�ado para estimar los par�metros de los modelos planteados. En el caso de ARIMA hay varios paquetes que soportan la estimaci�n de los modelos, el an�lisis de los resultados y la elaboraci�n de los pron�sticos. A prop�sito del software usado al trabajar con modelos estado-espacio \cite{Commandeur2011} presenta una revisi�n de los principales programas y paquetes, dentro de los que incluye: 

% Es posible copiar la tabla completa. 
 
\begin{itemize}
\item Eview
\item gretl
\item MATLAB
\item R base
\item R + KFAS
\item RATS
\item REGCMPNT 
\item SAS
\item S-PLUS
\item SsfPack
\item SsfPack
\item STAMP
\item Stata
\end{itemize}

Para mantener un �nico ambiente de desarrollo, nos enfocamos en los paquetes desarrollados en R, como los mencionados por Petris \cite{Petris2011}, Helske \cite{Helske2017} y Tusell \cite{Tusell2011}


\section{Identificaci�n del modelo}

\section{Simulaci�n Monte-Carlo}

\section{Estimaci�n de par�metros}

\section{Proyecciones 2017 - 2020}

\section{Conclusiones}