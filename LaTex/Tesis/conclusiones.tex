\chapter{Conclusiones}

Las poblaciones carcelarias como otras poblaciones peque�as est�n  ligadas a la poblaci�n total, y a su distribuci�n por edad  y g�nero; en este caso existen tasas de encarcelamiento diferenciales por edad y g�nero que influyen en el comportamiento de la poblaci�n carcelaria en el largo plazo. Una parte del incremento de la poblaci�n carcelaria total, puede ser asociado al crecimiento de la poblaci�n nacional y al cambio de su estructura por edad. En este sentido es importante incluir informaci�n de la poblaci�n total, al proyectar poblaciones carcelarias, particularmente en el largo plazo.

La calidad y la disponibilidad de los datos pueden hacer dif�cil incluir en las proyecciones informaci�n de la poblaci�n total. En el desarrollo de este documento solo se cont� con dos a�os de informaci�n sobre la poblaci�n carcelaria separada por edad, en la granularidad requerida. En casos como este, donde se cuenta con conteos de poblaci�n en periodos m�s extensos de tiempo es posible proyectar estos conteos como SARIMA. Trabajar con modelos SARIMA permite estimar los componentes estacionales, para mejorar la precisi�n. Los modelos ajustados sugieren un buen desempe�o de la proyecci�n en el corto plazo.

Al contar con informaci�n sobre los conteos de poblaci�n sindicada y condenada es posible modelar la interacci�n entre estas, para tener pron�sticos m�s precisos, especialmente si esta proyecci�n se puede realizar por delito. Trabajar por delito permite identificar cambios estructurales asociados a cambios en la pol�tica criminal, que no son f�cilmente observables en el total de la poblaci�n privada de la libertad.

Uno de los usos de las proyecciones de poblaciones peque�as es la proyecci�n de la infraestructura necesaria para atender las necesidades de la poblaci�n en cuesti�n. En el caso de poblaciones privadas de la libertad, el uso de infraestructuras separadas hace que sea cr�tico realizar la proyecci�n separada por g�nero. Dado que los delitos por los que ingresan son diferentes entre hombres y mujeres, esta proyecci�n deber�a considerar proyecciones separadas por delito. 

Los trabajos futuros sobre poblaciones carcelarias en Colombia estar�n mediadas por la calidad y la disponibilidad de datos sobre esta poblaci�n.